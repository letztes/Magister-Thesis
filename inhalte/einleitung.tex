%%%%%%%%%%%%%%%%%%%%%%%%%%%%%%%%%%%%%%%%%%%%%%%%%%%%%%%%%%%%%%%%
%% Vorlage fuer akademische Arbeiten, Gestaltungsrichtlinien der
%% Medienwissenschaft an der Universitaet Regensburg
%% 
%% Gutes Gelingen  --  Juli 2008
%% Christoph Mandl und Christoph Pfeiffer
%% http://www-mw.uni-r.de/studium/materialien 
%%
%% CC/BY-SA/3.0 - http://creativecommons.org/licenses/by-sa/3.0/
%%
%%%%%%%%%%%%%%%%%%%%%%%%%%%%%%%%%%%%%%%%%%%%%%%%%%%%%%%%%%%%%%%%

\cleardoublepage
\phantomsection

%\thispagestyle{section}
\abstand
\addcontentsline{toc}{section}{Einleitung}
\section*{Einleitung}

Die Syntax ist eine etablierte sprachwissenschaftliche Disziplin mit langer Tradition und hat dementsprechend eine F�lle von Modellen entwickelt, die sich teilweise erg�nzen, teilweise in unvereinbarer Konkurrenz zueinander stehen\footnote{\a{[...] es ist kaum zu bestreiten, dass es -- beim heutigen Stand unserer sprachwissenschaftlichen Erkenntnisse -- nicht nur vorkommt, dass zwei (oder mehr) Theorien dieselben fakten [sic] erkl�ren (und auch weitgehend ineinander �bersetzt werden k�nnen) oder dass eine Theorie prizipiell ad�quater ist als die andere, sondern dass oft auch zu beobachten ist, dass eine Theorie A die Sachverhalte a und b besser erkl�rt als eine Theorie B, die ihrerseits die Sachverhalte c und d besser zu erkl�ren vermag als die Theorie A} \citep[88]{Helbig1995}}. Zu den wichtigsten Vertretern der Syntaxforschung geh�ren zweifellos Noam Chomsky und Lucien Tesni�re.

Chomskys Konzept der Universalgrammatik als angeborenes kognitives Muster eines jeden Menschen bildet das Fundament f�r eine Reihe von Theorien und Untertheorien, die von zahlreichen Wissenschaftlern �ber die Zeit entwickelt und weitergef�hrt wurden. Dieses Theoriegeb�ude ist sehr abstrakt und wendet die Fragestellung von der Beschreibung der Struktur der �u�erung hin zur Beschreibung der �u�erungsproduktion und den zu Grunde liegenden Prozessen. Die �berschneidung zur Psychologie und Kognitionsforschung ist gro� und verleiht der Forschungsrichtung einen interdisziplin�ren Charakter. 

Tesni�res Dependenzgrammatik besch�ftigt sich mit der Struktur der �u�erung und gewann besonders in der p�dagogischen Anwendung eine nicht zu untersch�tzende Bedeutung. In kaum einem Einf�hrungswerk zur Linguistik fehlt ein Kapitel zur Dependenzgrammatik. �ber die Zeit entwickelte sich ein Geflecht von Formalismen, die auf dem Dependenz- und Valenzmodell beruhen, wie z.~B. die Tiefenkasustheorie oder die Kopfgesteuerte Phrasenstrukturgrammatik HPSG (Head Driven Phrase Structure Grammar).

Tesni�re z�hlte zu den ersten Mitgliedern des Prager Linguistenzirkels \citep[17]{Tesniere1980} und die sowjetische Akademiegrammatik \citep[13-82]{Svedova1980} stellt das Valenz- und Dependenzkonzept\footnote{Dort hei�t es jedoch \a{\rus{Podqinitelp1nye} \rus{svj1zi} \rus{slov} \rus{i} \rus{slovosoqetanij1}}; nach dem Begriff \rus{valentnostp1} sucht man vergebens, ebensowenig findet man Tesni�res Stemmata.} als Grundlage aller weiteren �berlegungen dar. Igor' Mel'\v{c}uks Arbeiten zum Text-Bedeutungsmodell \citep{Kahane2006} beinhalten Teile sowohl der Generativen Transformationsgrammatik als auch der Dependenzgrammatik.
In der russistischen Sprachwissenschaft als Ganzes betrachtet hat sich die Dependenzgrammatik aber nicht in der Form etablieren k�nnen so wie sie es etwa in der germanistischen Forschung getan hat.
\begin{zitat}
    Ein einheitlicher und methodisch wie theoretisch im Detail ausgearbeiteter Valenzbegriff ist [...] in der russistischen Sprachwissenschaft nicht in Sicht. \citep[1210]{Nuebler2006}
\end{zitat}
Dies verwundert.

Die vorliegende Arbeit stellt einen Versuch dar, zu er�rtern, wie gut sich die Dependenzgrammatik auf das Russische anwenden l�sst. Diese Frage erwuchs von einem computerlinguistischen Standpunkt, denn w�hrend meiner bisherigen Besch�ftigung mit kommerzieller Dependenz-Parsersoftware \citep{Spengler2008} stellte ich fest, dass gerade f�r das Russische Implementierungen fehlen. 
%Diese Beobachtung widerspricht der oben erw�hnten Durchdringung der Valenz- und Dependenzidee in den Grammatographien.
Die Frage nach der Anwendbarkeit ist unter anderem dadurch motiviert; sie d�rfte aber auch in anderen Kontexten interessant sein.
