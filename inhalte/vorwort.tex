%%%%%%%%%%%%%%%%%%%%%%%%%%%%%%%%%%%%%%%%%%%%%%%%%%%%%%%%%%%%%%%%
%% Vorlage fuer akademische Arbeiten, Gestaltungsrichtlinien der
%% Medienwissenschaft an der Universitaet Regensburg
%% 
%% Gutes Gelingen  --  Juli 2008
%% Christoph Mandl und Christoph Pfeiffer
%% http://www-mw.uni-r.de/studium/materialien 
%%
%% CC/BY-SA/3.0 - http://creativecommons.org/licenses/by-sa/3.0/
%%
%%%%%%%%%%%%%%%%%%%%%%%%%%%%%%%%%%%%%%%%%%%%%%%%%%%%%%%%%%%%%%%%

\cleardoublepage
\phantomsection

%\thispagestyle{section}
\abstand
\addcontentsline{toc}{section}{Vorwort}
\section*{Vorwort}

Die vorliegende Arbeit gliedert sich grob in zwei Teile, einen theoretischen und einen praktischen. Im ersten Teil f�hre ich in das Gebiet der Syntax ein und stelle neben anderen die Dependenzgrammatik vor, die ich auf knappem Raum mit der Konstituentenstrukturgrammatik und mit der Generativen Grammatik vergleiche. Bei den Ausf�hrungen zur Dependenzgrammatik st�tze ich mich im Wesentlichen auf Tesni�res Hauptwerk \citep{Tesniere1980}, erg�nzt durch neuere Arbeiten zum selben Thema. Im praktischen Teil wende ich die Dependenzgrammatik auf konkrete russischer S�tze an, die ich einem aktuellen Zeitungsartikel entnommen habe. Es handelt sich dabei um die Onlineausgabe der Zeitung \rus{Novaya Gazeta} \citep{Nikitinskiy2010}. Der Artikel ist im Anhang sowohl als Reintext als auch in einer dependenzgrammatisch annotierten Version zu finden.

Ich verzichte darauf, an dieser Stelle anzurei�en, was die einzelnen Kapitel zum Gegenstand haben; ich habe mir M�he gegeben, aussagekr�ftige �berschriften zu w�hlen, sodass der Leser anhand des Inhaltsverzeichnisses sich ein Bild davon machen kann.

Zuletzt noch ein Hinweis zu Quellenangaben und Fu�noten:\\
Literaturverweise habe ich stets im Flie�text direkt an der Stelle untergebracht, wo sich der Bezug zur Quelle befindet. Fu�noten habe ich verwendet, um zus�tzliche, f�r den Kern der Sache zwar entbehrliche, f�r das Verst�ndnis des Themas jedoch f�rderliche Informationen anzugeben. Es erscheint mir sinnvoll, derlei aus dem laufenden Text auszulagern, da es den Satz oder Absatz zu sehr aufbl�hen und die R�ckkehr aus dem Exkurs in den Lesefluss zu abrupt gestalten w�rde. Die Fu�noten k�nnen �bergangen werden, wenn der Leser es eilig hat.
